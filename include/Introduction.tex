% CREATED BY DAVID FRISK, 2016
\chapter{Introduction}
The number of devices connected to the Internet has grown impressively for some time and it is predicted that by 2020 there will be around 50 billion devices connected to the Internet \citep{BELLO201752}.  More and more of these devices are physical objects, such as sensors and actuators. This phenomenon is known as the Internet of Things (IoT).

There are many unsolved problems in the field of Wireless Sensor Networks (WSNs). One such problem is the complexity of designing, coding, and testing Wireless Sensor Network (WSN) applications. The problem arises because of the distributed environment of heterogeneous devices in a WSN, where each device may have limited resources and the communication between devices may have many specific requirements, e.g. the full TCP/IP stack can be missing \citep{sivieri2012wsn}.

One possible approach to solve some of these problems is to use the Erlang language which was developed in the 1980s by Ericsson. At first Erlang was developed for embedded telecommunication systems but it has over time continued to grow and now it is a complete platform with many libraries that offer functionality for a wide range of different applications. 

The Erlang language was developed as a high-level functional language for hiding distribution with facilities like:

\begin{itemize}
    \item Support for building software with guaranteed fault tolerance.
    \item Support for binaries as a data type which in turn allows pattern matching on bit-streams.
    \item A lightweight concurrency model
    \item Solving the problem of reusability because of heterogeneous devices in a WSN by the use of a virtual machine.
    \item Distributed programming, with support for high-level communication primitives. 
    \item Support for transparent resolution of process names over a network.
\end{itemize}

Because of all these points it should be possible to achieve reliable and fault-tolerant communication between nodes in a WSN and because of this Erlang seems like a good fit for a developing WSN applications \citep{sivieri2012wsn, sivieri2012erlang}.

\section{Internet of Things}\label{sec:intro_iot}
The concept of IoT is to connect every network-enabled device to the Internet to create a “smart world”, where our everyday objects can connect to each other to share data with the aim of enhancing our lives. There are many examples of applications that can enhance life in areas such as healthcare, smart buildings, social networks, environment monitoring, transportation and logistics, etc. Every IoT application depends on collected data from a network-enabled device or devices and there exist many different data collection devices and systems, e.g. RFID, sensors, wireless sensor networks (WSNs) \citep{yang2014internet}.

\section{Aim}
The goal of this thesis is to evaluate the performance of using Erlang to manage the communication in a WSN. To achieve this evaluation of the performance two prototypes will be developed, One with the low-level language C++ and the other one with the high-level language Erlang. The performance is going to be measured by comparing the number of lines of code, to investigate the effort of implementation for the different prototypes, and the utilization of CPU, RAM and power for different rates of sending messages.

In a WSN it is often required that the devices are power efficient and that devices can communicate directly with each other. This is something that the standard Erlang distribution does not implement and a connection to a suitable networking protocol that can manage device to device communication must therefore also be implemented as part of the thesis.

For the resulting thesis to be considered successful the following points should be achieved:

\begin{itemize} 
    \item A C++ prototype that uses a suitable network protocol for device to device communication and energy efficiency. 
    \item An Erlang prototype that uses the same protocol as the C++ prototype. Because of this some connection to this protocol must be implemented as part of the Erlang prototype. 
    \item An analysis of data collected from the prototypes presented so as to give someone who want to develop the communication between nodes in a WSN with Erlang a guideline for the hardware requirements.   
\end{itemize}

\section{Limitations}
There are many network protocols and multiple variants of some of them. Because of the limited time for the project only one of these protocols and variants is going to be investigated. 

\section{Affiliation}
The thesis work was carried out at the company Cipherstone Technologies AB. The company is currently in the first stages of developing a new WSN product. This product is planned to be a video-based sensor network and they want to investigate communication solutions for the network.



