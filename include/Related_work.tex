\chapter{Related work}
By approaching the work in this thesis from different direction there are multiple areas with related works. In this section related works are going to be mentioned about how to modify the Erlang runtime system for WSN application, comparisons between Erlang based languages and other languages, how functional languages can reduce security risks, different device-to-device communication techniques, the power consumption of ZigBee, and how a functional language can be used to generate nesC code. 

In the papers \citet{sivieri2012wsn} and \citet{sivieri2016building} the approach was to investigate how the Erlang runtime system could be modified for WSN application. In \citet{sivieri2012wsn} they write about how WSN-Erlang gives a higher level of programming abstraction to make it easier to produce more reusable, maintainable code and make it easier to test code that may run on heterogenous networks. They also mention how they have striped the runtime system of unnecessary libraries and facilities to reduce the memory, storage and processing requirements. 

\citet{sivieri2016building} is in some ways a continuation of the work done in \citet{sivieri2012wsn}. The development platform ELIoT is presented and a comparison is preformed between a C implementation, the framework AllJoyn paired with Java implementation and an ELIoT implementation of a smart-home application. The conclusions they draw from these comparisons is that ELIoT make it possible to develop more concise and readable code that is easier to test and debug. They also show that the memory usage between ELIoT and a C implementation is comparable and that the CPU usage still is within practical limits for an ELIoT implementation of the same C application. 

In \citet{fedrecheski2016elixir} another performance comparison is preformed between a Swarm Broker implementation in Java and Elixir. Elixir is a language with the goal to leverage all abstractions of Erlang, whilst at the same time add new features from other programming languages \citep{thomas2018programming}. The conclusions that they draw in this paper is that the Java implementation of the Swarm Broker application require slightly less CPU usage. However, Elixir shows better memory usage and the number of lines of code is markedly less. 

Another direction to approach the work in this thesis is to approach from the direction of the security of an application. In \citet{haenisch2016case} a case study is preformed that shows that the use of a functional language or functional techniques can reduce the code size and the complexity of an application. From this follows that the code of an application has less bugs and therefore less security problems.

In this thesis ZigBee was used for device-to-device communication but there exist alternatives to this solution. In \citet{essameldindevice} and \citet{militano20155g} different techniques are investigated for the task of device-to-device communications in the case of IoT. The fifth generation (5G) cellular system is also discussed in \citet{militano20155g} for the main challenges and the coming research directions that needs to be investigated to reach what is expected to be a reality for IoT, a device-oriented Anything-as-a-Service ecosystem.

The problem in this thesis is to investigate how a device in a WSN using Erlang and communicating over ZigBee preforms against a low-level implementation of the same application. Because of this it could be interesting to investigate the performance of IEEE 802.15.4 LR-WPAN. In \citet{kohvakka2006performance} the IEEE 802.15.4 standard MAC protocol is analysed to investigate the network performance and energy efficiency.

\citet{mainland2008flask} is a related work that shows that there are other approaches than the one presented in this thesis to make it possible to use functional programming for devices with a constrained availability of resources in a WSN scenario. The paper presents Flask, a domain specific language embedded in Haskell that generate nesC (nesC is a dialect of C) code for TinyOS.
